\documentclass[12pt,letterpaper]{hmcpset}
\usepackage[margin=1in]{geometry}
\usepackage{graphicx, cancel}
\usepackage{amsthm}
\usepackage{enumitem}
\usepackage{amsmath}
\usepackage{changepage}
\usepackage{breqn}
\setlength{\parindent}{0 pt}
\setlength{\parskip}{1 em}

\usepackage{hyperref}
\hypersetup{
    colorlinks=true,
    linkcolor=blue,
    filecolor=magenta,      
    urlcolor=cyan,
}
 
% Theorems
\usepackage{amsthm}
\renewcommand\qedsymbol{$\blacksquare$}
\makeatletter
\@ifclassloaded{article}{
    \newtheorem{definition}{Definition}[section]
    \newtheorem{example}{Example}[section]
    \newtheorem{theorem}{Theorem}[section]
    \newtheorem{corollary}{Corollary}[theorem]
    \newtheorem{lemma}{Lemma}[theorem]
}{
}
\makeatother

% Random Stuff
\setlength\unitlength{1mm}

\newcommand{\insertfig}[3]{
\begin{figure}[htbp]\begin{center}\begin{picture}(120,90)
\put(0,-5){\includegraphics[width=12cm,height=9cm,clip=]{#1}}\end{picture}\end{center}
\caption{#2}\label{#3}\end{figure}}

\newcommand{\insertxfig}[4]{
\begin{figure}[htbp]
\begin{center}
\leavevmode \centerline{\resizebox{#4\textwidth}{!}{\input
#1.pstex_t}}
\caption{#2} \label{#3}
\end{center}
\end{figure}}

\long\def\comment#1{}

\newcommand\abs[1]{\left\lvert#1\right\rvert}
\newcommand\norm[1]{\left\lVert#1\right\rVert}
\DeclareMathOperator*{\argmin}{arg\,min}
\DeclareMathOperator*{\argmax}{arg\,max}

% bb font symbols
\newfont{\bbb}{msbm10 scaled 700}
\newcommand{\CCC}{\mbox{\bbb C}}

\newfont{\bbf}{msbm10 scaled 1100}
\newcommand{\CC}{\mbox{\bbf C}}
\newcommand{\PP}{\mbox{\bbf P}}
\newcommand{\RR}{\mbox{\bbf R}}
\newcommand{\QQ}{\mbox{\bbf Q}}
\newcommand{\ZZ}{\mbox{\bbf Z}}
\renewcommand{\SS}{\mbox{\bbf S}}
\newcommand{\FF}{\mbox{\bbf F}}
\newcommand{\GG}{\mbox{\bbf G}}
\newcommand{\EE}{\mbox{\bbf E}}
\newcommand{\NN}{\mbox{\bbf N}}
\newcommand{\KK}{\mbox{\bbf K}}
\newcommand{\KL}{\mbox{\bbf KL}}

% Vectors
\renewcommand{\aa}{{\bf a}}
\newcommand{\bb}{{\bf b}}
\newcommand{\cc}{{\bf c}}
\newcommand{\dd}{{\bf d}}
\newcommand{\ee}{{\bf e}}
\newcommand{\ff}{{\bf f}}
\renewcommand{\gg}{{\bf g}}
\newcommand{\hh}{{\bf h}}
\newcommand{\ii}{{\bf i}}
\newcommand{\jj}{{\bf j}}
\newcommand{\kk}{{\bf k}}
\renewcommand{\ll}{{\bf l}}
\newcommand{\mm}{{\bf m}}
\newcommand{\nn}{{\bf n}}
\newcommand{\oo}{{\bf o}}
\newcommand{\pp}{{\bf p}}
\newcommand{\qq}{{\bf q}}
\newcommand{\rr}{{\bf r}}
\renewcommand{\ss}{{\bf s}}
\renewcommand{\tt}{{\bf t}}
\newcommand{\uu}{{\bf u}}
\newcommand{\ww}{{\bf w}}
\newcommand{\vv}{{\bf v}}
\newcommand{\xx}{{\bf x}}
\newcommand{\yy}{{\bf y}}
\newcommand{\zz}{{\bf z}}
\newcommand{\0}{{\bf 0}}
\newcommand{\1}{{\bf 1}}

% Matrices
\newcommand{\Ab}{{\bf A}}
\newcommand{\Bb}{{\bf B}}
\newcommand{\Cb}{{\bf C}}
\newcommand{\Db}{{\bf D}}
\newcommand{\Eb}{{\bf E}}
\newcommand{\Fb}{{\bf F}}
\newcommand{\Gb}{{\bf G}}
\newcommand{\Hb}{{\bf H}}
\newcommand{\Ib}{{\bf I}}
\newcommand{\Jb}{{\bf J}}
\newcommand{\Kb}{{\bf K}}
\newcommand{\Lb}{{\bf L}}
\newcommand{\Mb}{{\bf M}}
\newcommand{\Nb}{{\bf N}}
\newcommand{\Ob}{{\bf O}}
\newcommand{\Pb}{{\bf P}}
\newcommand{\Qb}{{\bf Q}}
\newcommand{\Rb}{{\bf R}}
\newcommand{\Sb}{{\bf S}}
\newcommand{\Tb}{{\bf T}}
\newcommand{\Ub}{{\bf U}}
\newcommand{\Wb}{{\bf W}}
\newcommand{\Vb}{{\bf V}}
\newcommand{\Xb}{{\bf X}}
\newcommand{\Yb}{{\bf Y}}
\newcommand{\Zb}{{\bf Z}}

% Calligraphic
\newcommand{\Ac}{{\cal A}}
\newcommand{\Bc}{{\cal B}}
\newcommand{\Cc}{{\cal C}}
\newcommand{\Dc}{{\cal D}}
\newcommand{\Ec}{{\cal E}}
\newcommand{\Fc}{{\cal F}}
\newcommand{\Gc}{{\cal G}}
\newcommand{\Hc}{{\cal H}}
\newcommand{\Ic}{{\cal I}}
\newcommand{\Jc}{{\cal J}}
\newcommand{\Kc}{{\cal K}}
\newcommand{\Lc}{{\cal L}}
\newcommand{\Mc}{{\cal M}}
\newcommand{\Nc}{{\cal N}}
\newcommand{\Oc}{{\cal O}}
\newcommand{\Pc}{{\cal P}}
\newcommand{\Qc}{{\cal Q}}
\newcommand{\Rc}{{\cal R}}
\newcommand{\Sc}{{\cal S}}
\newcommand{\Tc}{{\cal T}}
\newcommand{\Uc}{{\cal U}}
\newcommand{\Wc}{{\cal W}}
\newcommand{\Vc}{{\cal V}}
\newcommand{\Xc}{{\cal X}}
\newcommand{\Yc}{{\cal Y}}
\newcommand{\Zc}{{\cal Z}}

% Bold greek letters
\newcommand{\alphab}{\hbox{\boldmath$\alpha$}}
\newcommand{\betab}{\hbox{\boldmath$\beta$}}
\newcommand{\gammab}{\hbox{\boldmath$\gamma$}}
\newcommand{\deltab}{\hbox{\boldmath$\delta$}}
\newcommand{\etab}{\hbox{\boldmath$\eta$}}
\newcommand{\lambdab}{\hbox{\boldmath$\lambda$}}
\newcommand{\epsilonb}{\hbox{\boldmath$\epsilon$}}
\newcommand{\nub}{\hbox{\boldmath$\nu$}}
\newcommand{\mub}{\hbox{\boldmath$\mu$}}
\newcommand{\zetab}{\hbox{\boldmath$\zeta$}}
\newcommand{\phib}{\hbox{\boldmath$\phi$}}
\newcommand{\psib}{\hbox{\boldmath$\psi$}}
\newcommand{\thetab}{\hbox{\boldmath$\theta$}}
\newcommand{\taub}{\hbox{\boldmath$\tau$}}
\newcommand{\omegab}{\hbox{\boldmath$\omega$}}
\newcommand{\xib}{\hbox{\boldmath$\xi$}}
\newcommand{\sigmab}{\hbox{\boldmath$\sigma$}}
\newcommand{\pib}{\hbox{\boldmath$\pi$}}
\newcommand{\rhob}{\hbox{\boldmath$\rho$}}

\newcommand{\Gammab}{\hbox{\boldmath$\Gamma$}}
\newcommand{\Lambdab}{\hbox{\boldmath$\Lambda$}}
\newcommand{\Deltab}{\hbox{\boldmath$\Delta$}}
\newcommand{\Sigmab}{\hbox{\boldmath$\Sigma$}}
\newcommand{\Phib}{\hbox{\boldmath$\Phi$}}
\newcommand{\Pib}{\hbox{\boldmath$\Pi$}}
\newcommand{\Psib}{\hbox{\boldmath$\Psi$}}
\newcommand{\Thetab}{\hbox{\boldmath$\Theta$}}
\newcommand{\Omegab}{\hbox{\boldmath$\Omega$}}
\newcommand{\Xib}{\hbox{\boldmath$\Xi$}}

% mixed symbols
\newcommand{\sinc}{{\hbox{sinc}}}
\newcommand{\diag}{{\hbox{diag}}}
\renewcommand{\det}{{\hbox{det}}}
\newcommand{\trace}{{\hbox{tr}}}
\newcommand{\tr}{\trace}
\newcommand{\sign}{{\hbox{sign}}}
\renewcommand{\arg}{{\hbox{arg}}}
\newcommand{\var}{{\hbox{var}}}
\newcommand{\cov}{{\hbox{cov}}}
\renewcommand{\Re}{{\rm Re}}
\renewcommand{\Im}{{\rm Im}}
\newcommand{\eqdef}{\stackrel{\Delta}{=}}
\newcommand{\defines}{{\,\,\stackrel{\scriptscriptstyle \bigtriangleup}{=}\,\,}}
\newcommand{\<}{\left\langle}
\renewcommand{\>}{\right\rangle}
\newcommand{\Psf}{{\sf P}}
\newcommand{\T}{\top}
\newcommand{\m}[1]{\begin{bmatrix} #1 \end{bmatrix}}


\DeclareMathOperator{\atan}{atan}
\DeclareMathOperator{\acos}{acos}
\DeclareMathOperator{\R}{\mathbb{R}}
% info for header block in upper right hand corner
\name{-----------------------------------------}
\class{Differential Geometry}
\assignment{Homework 7}
\duedate{Monday, November 11 2019}

\renewcommand{\labelenumi}{{(\alph{enumi})}}

\begin{document}

\begin{problem}[A.a) Write up a proof for the Key Theorem on page 216, Baby Do
Carmo.]
\end{problem}
\begin{solution}
\end{solution}

\medskip

\newpage
\begin{problem}[B.a) Carry out the details for Example 5, page 162, Baby Do
  Carmo.  (Including the application to a geometric interpretation 
of the Dupin indicatrix, that is from page 164 to 165, Baby Do Carmo.)]
\end{problem}
\begin{solution}
\end{solution}

\newpage
\begin{problem}[C.a) Problem 2 on page 151, Section 3-2, Baby Do Carmo.]
\\ \\
Show that if a surface is tangent to a plane along a curve, then the points of this curve are either parabolic or planar
\end{problem}
\begin{solution}
  Let $S$ be the surface, let $N(p)$ be the orientation at a point $p \in T_p(S)$.
Let the curve be $\alpha(t)$. This curve is in the intersection of our surface and
  our plane. For any point $\alpha(t_0)$ on the curve the tangent plane
  is always the same.  Therefore $dN(\alpha(t_0))_{\alpha'(t_0)} = \vec{0}$. Then let
  $a=\frac{\alpha''(t_0)}{||\alpha''(t_0)||}$ . So $a$ is a unit vector in $T_{\alpha(t_0)}(S)$
  perpindicular to $\alpha'(t_0)$. Then let $b = dN(\alpha(t_0))_{a}$. Then gaussian
  curvature is $det(dN(\alpha(t_0))) = 0$. Therefore, the curve is either parabolic
  or planar. 
\end{solution}

\newpage
\begin{problem}[C.b) Problem 6 on page 151, Section 3-2, Baby Do Carmo.]
\\ \\
Show that the sum of the normal curvatures for any pair of orthogonal directions, at a point $p\in S$, is constant.
\end{problem}
\begin{solution}
 Let $\mathbf{x}_u, \mathbf{x}_v \in T_p(S)$ be an orthonormal basis for $T_p(S)$.
Let $a, b = a_1\mathbf{x}_u + a_2\mathbf{x}_v, b_1\mathbf{x}_u + b_2\mathbf{x}_v
\in T_p(S)$ be a pair of orthogonal vectors.
%Assume without loss of generality that
%they are unit length since normal curvature is invariant under lengths of the
%vectors.  Let $N_p(v), v \in T_p(S)$ be the gauss map. Let $c = N_p(0)$. Then $b = c \land a$. We use second fundamental
%form to compute their normal curvatures. Since $a, b$ are orthogonal,
%$a_1b_1<\mathbf{x}_u, \mathbf{x}_u> + a_1b_2<\mathbf{x}_u, \mathbf{x}_v> +
%a_2b_1<\mathbf{x}_u, \mathbf{x}_v> + a_2^2<\mathbf{x}_v, \mathbf{x}_v> = 0$.
The sum of their normal curvatures
is, \\
\begin{align*}
  -<dN_p(a), a> - <dN_p(b), b> &= -<a_1dN_p(\mathbf{x}_u) + a_2
                                 dN_p(\mathbf{x}_v), a_1\mathbf{x}_u + a_2\mathbf{x}_v> \\
                               & -<b_1dN_p(\mathbf{x}_u) + b_2
                                 dN_p(\mathbf{x}_v), b_1\mathbf{x}_u + b_2\mathbf{x}_v> \\
                               = -(a_1^2&<dN_p(\mathbf{x}_u), \mathbf{x}_u> + \\
                                 a_1a_2&<dN_p(\mathbf{x}_u), \mathbf{x}_v> + \\
                                 a_2a_1&<dN_p(\mathbf{x}_v), \mathbf{x}_u> + \\
                                 a_2^2 &<dN_p(\mathbf{x}_v), \mathbf{x}_v> + \\
                                 b_1^2 &<dN_p(\mathbf{x}_u), \mathbf{x}_u> + \\
                                 b_1b_2&<dN_p(\mathbf{x}_u), \mathbf{x}_v> + \\
                                 b_2b_1&<dN_p(\mathbf{x}_v), \mathbf{x}_u> + \\
                                 b_2^2 &<dN_p(\mathbf{x}_v), \mathbf{x}_v>
                                         ) \\
                              = -((a_1^2 + b_1^2)<dN_p(\mathbf{x}_u), \mathbf{x}_u> + 
                                 2(a_1a_2 + b_1b_2)&<dN_p(\mathbf{x}_u), \mathbf{x}_v> + 
                                                     (a_2^2 + b_2^2)<dN_p(\mathbf{x}_v), \mathbf{x}_v>) \\
  = -((a_1^2 + a_2^2)<dN_p(\mathbf{x}_u), \mathbf{x}_u> + 
                                 2(\cancel{a_1a_2 - a_2a_1})&<dN_p(\mathbf{x}_u), \mathbf{x}_v> + 
                                                     (a_2^2 + a_1^2)<dN_p(\mathbf{x}_v), \mathbf{x}_v>) \\
 = -(<dN_p(\mathbf{x}_u), \mathbf{x}_u> + <dN_p(\mathbf{x}_v), \mathbf{x}_v>)
\end{align*}
So the sum of the curvatures does not depend on the pair chosen. 
  % &= 
%  -<dN_p(a), a> - <dN_p(b), b> &= -<dN_p(a), a> - <dN_p(c \land a), c \land a> \\
%                               &= -(<dN_p(a), a> + <dN_p(c) \land dN_p(a), c \land a>) \\
%                               &= -(<dN_p(a), a> + <dN_p(c), c> \cdot <dN_p(a), 
%                                 a> \\
%                               & -<dN_p(c), a> \cdot <dN_p(a), c>) \\
%  -<dN_p(a), a> - <dN_p(b), b> &= -<a_1dN_p(\mathbf{x}_u) + a_2
%                                 dN_p(\mathbf{x}_v), a_1\mathbf{x}_u + a_2\mathbf{x}_v> \\
%                               & -<b_1dN_p(\mathbf{x}_u) + b_2
%                                 dN_p(\mathbf{x}_v), b_1\mathbf{x}_u + b_2\mathbf{x}_v> \\
%                               = -(a_1^2&<dN_p(\mathbf{x}_u), \mathbf{x}_u> + \\
%                                 a_1a_2&<dN_p(\mathbf{x}_u), \mathbf{x}_v> + \\
%                                 a_2a_1&<dN_p(\mathbf{x}_v), \mathbf{x}_u> + \\
%                                 a_2^2 &<dN_p(\mathbf{x}_v), \mathbf{x}_v> + \\
%                                 b_1^2 &<dN_p(\mathbf{x}_u), \mathbf{x}_u> + \\
%                                 b_1b_2&<dN_p(\mathbf{x}_u), \mathbf{x}_v> + \\
%                                 b_2b_1&<dN_p(\mathbf{x}_v), \mathbf{x}_u> + \\
%                                 b_2^2 &<dN_p(\mathbf{x}_v), \mathbf{x}_v>
%                                         ) \\
%                              = -((a_1^2 + b_1^2)<dN_p(\mathbf{x}_u), \mathbf{x}_u> + 
%                                 2(a_1a_2 + b_1b_2)&<dN_p(\mathbf{x}_u), \mathbf{x}_v> + 
%                                 (a_2^2 + b_2^2)<dN_p(\mathbf{x}_v), \mathbf{x}_v>)
\end{solution}
\newpage
\begin{problem}[C.c) Problem 8 on page 151, Section 3-2, Baby Do Carmo.]
\\ \\
Desribe the region of the unit sphere covered by the Gauss map of the following surfaces:  
\begin{enumerate}
    \item Paraboloid of revolution $z=x^2+y^2$
    \item Hyperboloid of revolution $x^2+y^2 -z^2 = 1$
    \item Catenoid $x^2+y^2 = cosh^2z$
\end{enumerate}
\end{problem}
\begin{solution}
  \begin{enumerate}
  \item The bottom hemisphere
  \item the entire sphere
  \end{enumerate}
\end{solution}

\newpage
\begin{problem}[C.d) Problem 17 on page 152, Section 3-2, Baby Do Carmo.]
\\ \\
Show that if $H\equiv 0$ on $S$ and $S$ has no planar points, then the Gauss map $N:S\to S^2$ has the following property
$$\langle dN_p(w_1), dN_p(w_2)\rangle = -K(p)\langle w_1,w_2\rangle,\; \text{for all}\;p\in S \; \text{and all}\; w_1,w_2\in T_p(S)$$  
Show that the above condition implies that the angle of two intersecting curves on S and the angle of their spherical images (cf. Exercise 9) are equal up to a sign.\\
\\ 
For reference\\ \\
\textbf{Exercise 9:} Prove that 
\begin{enumerate}
    \item The image $N\circ \alpha$ by the Gauss map $N:S\to S^2$ of a parametrized regular curve $\alpha :I\to S$ which contains no planar or parabolic points is a parametrized regular curve on the sphere $S^2$ (called the \textit{spherical image} of $\alpha$).
    \item If $\alpha$ is a line of curvature, and $k$ is its curvature at $p$, then 
    $$k = |k_nK_N|$$
    where $k_n$ is the normal curvature at $p$ along the tangent line of $C$ and $k_N$ is the curvature of the spherical image $N(C)\subset S^2$ at $N(p)$.  
\end{enumerate}


\end{problem}
\begin{solution}
\end{solution}

\end{document}









